\documentclass[letterpaper, 24pt, final, onecolumn, titlepage] {article}

\usepackage{enumerate}
\usepackage{graphicx}
\usepackage{listings}
\usepackage{color}
\usepackage{setspace}
\usepackage {amsmath}
\usepackage{amssymb}

\title{ECE 270: Computer Methods in ECE \\
	\vspace{1.5cm}
   		\begin{center}\includegraphics{umlogo} \end{center}
	\vspace{1.5cm}
	\textbf{Assignment \#1} \\
	Math Review}
	
\author{Hussein El-Souri}

\date{\today}

\definecolor{dkgreen}{rgb}{0,0.6,0}
\definecolor{gray}{rgb}{0.5,0.5,0.5}
\definecolor{mauve}{rgb}{0.58,0,0.82}

\lstset{frame=tb,
  language=C,
  aboveskip=3mm,
  belowskip=3mm,
  showstringspaces=false,
  columns=flexible,
  basicstyle={\small\ttfamily},
  numbers=none,
  numberstyle=\tiny\color{gray},
  keywordstyle=\color{blue},
  commentstyle=\color{dkgreen},
  stringstyle=\color{mauve},
  breaklines=true,
  breakatwhitespace=true,
  tabsize=3
}

\begin{document}

\maketitle

\doublespacing

\section{Statement of the Problem}
The purpose of this assignment is to review mathmatical concepts and use \LaTeX\ to be able to produce those concepts into a technical report.\\

\pagebreak

\section{Description of Solution}
1.a) Suppose we have a straight line with points $(x_1, y_1) and (x_2, y_2)$.
In order to find the equation of the line first we should find the slope m such that $m=\frac{y_2-y_1}{x_2-x_1}$ and plug it into the equation:
\begin{equation} \label{line} 
	y - y_1 =  m(x - x_1)
\end{equation}
1.b) We should first rewrite the equation \ref{line} in slope intercept form as such:
\begin{equation} \label{line_intercept}
	y = mx + b
\end{equation}
And here we can replace the values of $y$ and $x$ with any desired value say $y^*$ and $x^*$ to get:
\begin{equation}\label{line_final}
y^* = mx^* + b
\end{equation}
2.The distance formula is the square root of the difference between the squares of the coordinates composing each point as such:
\begin{equation}\label{distance}
d = \sqrt{(x_2-x_1)^2 + (y_2-y_1)^2}
\end{equation}
\\\\3.a)We used desmos graphing calculator to draw any two random circles and label their centers and radii resulting in:
\begin{center}\includegraphics{circles} \end{center}
3.b) We can use the distance using the formula: $d = \sqrt{(x_2-x_1)^2 + (y_2-y_1)^2}$.\\
3.c)Two circles intersect if the distance between their centers is less than the sum of the radii as such: $d < (r_1 + r_2)$.\\\\
4.a) The general solution for a quadratic equation of the from$ax^2 + bx + c = 0$ is: $x_{1,2} = \frac{-b \pm \sqrt{b^2-4ac}}{2a}$.\\
\pagebreak
4.b) The discriminant d  is expressed as : $d = \sqrt{b^2-4ac}$.\\
4.c) The discriminant can be broken down into three possible solutions:
\begin{enumerate}
\item[i.] if $d>0$ we have two unique real solutions $x_{1,2} = \frac{-b \pm \sqrt{b^2-4ac}}{2a}$.
\item[ii.] if $d=0$ we have one repeated solution $x_1 = x_2 = \frac{-b}{2a}$.
\item[iii.] if $d<0$ we have two unique complex solutions $x_{1,2} = \frac{-b \pm i\sqrt{|b^2-4ac|}}{2a}$.\\
\end{enumerate}
5.a)First, we must understand how polar coordinates are represented. If we assume that each of those small circles are a point and connect the origin of the coordinate system to that point we form a radius and at angle $\theta$
Then take a perpendicular line from that point to the x-axis forming a right-angle triangle. Here we can use trigonometric identities to find that $x = r\cos\theta$ and $y = r\sin\theta$ as the image demonstrates:
\begin{center}\includegraphics{PolarCoordinateSystem} \end{center}
\pagebreak
Secondly, if we take look at the following picture:
\begin{center}\includegraphics{PolarCircles} \end{center}
We can see (still assuming each small circle is a point) that they small circles form a bigger circle with center at the origin.
A circle with radius $r$ and center$(h, k)$ as an equation of the general form:
\begin{equation}\label{circle}
(x-h)^2 + (y-k)^2 = r^2
\end{equation}
But in this case since the larger circle is centered at the origin $h = k = 0$ so the equation becomes:
\begin{equation} \label{circleOrigin}
x^2 + y^2 = r^2
\end{equation}
5.b) However; if the circle is not centered at the origin and is centered at $(cx, cy)$ then the values of $(h,k)$ in equation \ref{circle} will be replaced with the values of $(cx, cy)$ to get $(x-cx)^2 + (y-cy)^2 = r^2$.\\\\
6.a) The magnitude of a complex number $|z|$ is equal to the square root of the sum of the squares of its real part and imaginary part as such: $|z| = \sqrt{a^2 + b^2}$.\\
6.b)The Complex conjugate $z^*$ can be computed simply by inverting the sign of the imaginary part of the conjugate as such:$z^* = a - ib$.\\
6.c) The polar form of a complex number is as such $z= r( \cos\theta + i\sin\theta)$. \\
6.d) The sum $z$ of two complex numbers $z_1$ and $z_2$ is obtained by \\respectively adding the real parts and the imaginary parts of each as such: $z= (a_1 +a_2) + i(b_1 +b_2)$.
6.e) The product of two complex numbers is done through the distribution rule as such:
\begin{align*}
z &= z_1z_2\\
&= (a_1 +ib_1)(a_2 + ib_2)\\
&= a_1a_2 + ia_1b_2 + ib_1a_2 + i^2b_1b_2\\
&= a_1a_2 + ia_1b_2 + ib_1a_2 - b_1b_2\\
& \therefore\\
z &= ( a_1a_2 - b_1b_2) + i(a_1b_2 + b_1a_2)\\
\end{align*}

\pagebreak

\section{Testing and Output}
Displaying the pdf document correctly with no errors best ensures the success of the report written in \LaTeX.\\
The output is as expected but can be improved by possibly solving the assigned prooblems in functions that are already built into \LaTeX.\\
For example functions dedicated to representing complex numbers or polar coordinates. If these functions exist it would have decreased the number of lines/characters used to write this report.

\pagebreak

\end{document}