\documentclass[letterpaper, 24pt, final, onecolumn, titlepage] {article}

\usepackage{enumerate}
\usepackage{graphicx}
\usepackage{listings}
\usepackage{color}
\usepackage{setspace}

\title{ECE 270: Computer Methods in ECE \\
	\vspace{1.5cm}
   		\begin{center}\includegraphics{umlogo} \end{center}
	\vspace{1.5cm}
	\textbf{Assignment \#5} \\
	Average and Standard Deviation}
	
\author{Mary Jane}

\date{\today}

\definecolor{dkgreen}{rgb}{0,0.6,0}
\definecolor{gray}{rgb}{0.5,0.5,0.5}
\definecolor{mauve}{rgb}{0.58,0,0.82}

\lstset{frame=tb,
  language=C,
  aboveskip=3mm,
  belowskip=3mm,
  showstringspaces=false,
  columns=flexible,
  basicstyle={\small\ttfamily},
  numbers=none,
  numberstyle=\tiny\color{gray},
  keywordstyle=\color{blue},
  commentstyle=\color{dkgreen},
  stringstyle=\color{mauve},
  breaklines=true,
  breakatwhitespace=true,
  tabsize=3
}

\begin{document}

\maketitle

\doublespacing

\section{Statement of the Problem}

In this section, state \textbf{in your own words} what is the problem to be solved here. Focus on just the problem at hand and not the solution!

\section{Description of Solution}

In this section, explain how you solved the problem. Focus on the underlying marthematics and logic, rather than giving a line by line explanation of code. You can use brief code fragments in this section, but not large chunks of code. 

Break your writing into logical paragraphs! Do not write just one very long paragraph. Make your explanation have a logical flow; for example this needs to be explained before this, etc.

\section{Testing and Output}

In this section, explain what test you set up to verify that your program works; for example, what parameters were used. Give the output that the program produced along with some commentary. Is the output what you expected? How could it be improved?

\pagebreak

\section{Code}
\singlespacing

\begin{lstlisting}

#include <stdio.h>
#include <stdlib.h>

int main()
{
    printf("Hello world!\n");
    return 0;
}

\end{lstlisting}

\end{document}